\section{Introduction}

Twitter has already proven to be a powerful source of information for  monitoring
trends, sentiments and topics, 
%\cite{ramage2010characterizing}, 
but to
this date not a lot is known about Twitter users.  The majority of studies
regarding users focuses on issues such as the demographics of Twitter, 
%(including methods for inferring sex, age or location information)
%\cite{davisjr:2011}, 
community detection %\cite{java2007we} 
and measures to assess user influence.  However, only a 
few papers have focused on another very interesting issue: user profiling\cite{abel:2011}. 

A user profile is made of different elements, including user
preferences (as expressed in messages), personal information (such as gender and
location) or social network behavior (usage time and number of posts per day,
among others).  In contrast with previous methods for creating user profiles for
specific user accounts \cite{pennacchiotti2011democrats,abel2011semantic}, here
we are interested in generating profiles of \textit{groups of users}. By
working with groups of users we avoid problems of privacy and data scarcity
while being able to determine the underlying individual preferences \cite{zheleva:2009}. 

The user groups we want to profile comprise users that share at least one
common interest, which may become explicit when they follow the same Twitter
account or frequently discuss the same topic.  Identifying such groups and
profiles is of great value for companies, which can take advantage of them for
targeted marketing, recommendation, or to better understand
users preferences over time. On the other hand, this task is challenging because it may be seen as a clustering without a pre-defined similarity criterion.

%A few studies have previously proposed approaches to identify user preferences in social networks and microblogs \cite{pennacchiotti2011democrats,abel2011semantic}. 
%cho que a intro ficaria mais linear: redes sociais, user profiles, user groups, topico, termos, requisitos, upscale, contribuicoes, comentarios sobre os resultados (que podem ter que ser descartados


In this work, we define users' profiles based on their general preferences,
where a preference is defined by a \textit{semantic topic} discussed in their
posts. Identifying semantic topics is not a new task
\cite{pons2007topic}, although in many cases the
topic identification process is still based on the syntactic correlations
between terms and usually generates highly fragmented topics (or semantic
sub-topics). 

Formally, considering that we can extract semantic topics from users posts, any
set of users $U$ can be characterized by a weighted set of preferences
$<$$p_1$,$p_2$,$\ldots$,$p_n$$>$, where $p_x$ represents a semantic topic
identified from the posts of the group. Although initially we employ only
topics as preferences, other personal user characteristics may be added to
the preference set later.

In order to build users semantic profiles we propose the UPsCAle (User Profile
CreAtor) framework, which is organized in three main phases: (i) it identifies
semantic sub-topics of a user group using a traditional matrix factorization
method; (ii) it merges semantic sub-topics into more cohesive and unique
semantic topics; (iii) it maps the final set of semantic topics into users
profiles. The first phase uses a Non-negative Matrix Factorization (NMF)
\cite{berry2007algorithms} , ideal in this context since it does not assume that the latent
factors extracted from data compose a space of independent variables. This is
crucial, as topic overlapping might be common in the scenarios we are dealing
with.

Despite of their success in generating the so called semantic topics, matrix
factorization methods still generate latent factors that are fragmented (or
lack cohesion) and hence are not unique \cite{cheng:2013}. This results in latent factors
with different vocabularies often referring to the same topic. In order to deal
with the problem of topic fragmentation, in the second phase we propose a new
method for merging what we call semantic sub-topics (or latent factors) in
order to generate more cohesive and unique semantic topics. The proposed
method is based on Markovian theory and random walks, and works according to
the probability of topics reaching each other in a sub-topic transition graph.
Finally, we map users to topics according to the probabilities of their posts
belonging to a topic. 

This paper has two main contributions. The first, a new generic semantic topic
identification method that can be used in any topic detection task and
optimizes three contradictory desirable properties in topic identification
simultaneously: it finds the \textit{smallest representative} set of topics
\textit{highly cohesive} and \textit{non-fragmented} observed in a set of
messages. It also provides an efficient method for merging topics, which scales
to high volumes of posts. Finally, although intuitively, the length restriction
of posts in social media provides no information about context, the method
showed to overcome this problem.  The second contribution is a simple method to
define users profiles considering groups of users, represented by a weighted
set of preferences that initially correspond to semantic topics. 

\method was evaluated in two phases. First, we tested the topic identification
method in both text and social network collections where the main topics were
known. Second, we performed a case study with over 50,000 users that are
Twitter followers of Barack Obama, considering more than the 700,000 messages
posted during the American Elections in 2012. The results showed that the
method is able to generate less fragmented and more concise topics when
compared to other state of the art methods.% for topics identification.

%WMJ: do we have anything to say about users?

The remainder of this paper is organized as follows. Section 2 discusses related work in both user profiles and semantic topic identification. Section 3 introduces UPscALe, the framework developed to find users profiles according to semantic topics. Section 4 describes the experimental methodology, where tests wih both labeled and unlabeled data are performed. Finally, Section 5 draws conclusions and discusses future work directions.

