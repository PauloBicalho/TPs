\section{Conclusions and Future Work}

In this paper we analysed how the semantic topics can be used to extract
general characteristics from the users. 
We propose UPsCAle, a framework that works in three phases: it
identifies semantic sub-topics of a user group using a traditional matrix
factorization method and then merges those sub-topics into more cohesive
and unique semantic topics. In a last phase, it maps the final set of semantic
topics to users profiles.

The proposed method was evaluated considering two labelled dataset and an
unlabelled one. The results show that, when compared to other state of the art
methods for topic identification, \method can find better results than the baselines if the the trade-off between cohesion and uniqueness is
properly set. Furthermore, the method scales well to very large datasets.

As future works, the stopping criteria of \method will be carefully studied. As it generates a hierarchy of topics, identifying the level of interest to be explore is important. The method will also be enhanced with the approach proposed in \cite{cheng:2013} to deal better with data sparsity. Finally, other user characteristics will be added to the current profile.

%However, the same can be done with more spcific features. \gi{blabla}
