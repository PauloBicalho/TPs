\section{Introduction}

Twitter has already proven to be a powerful source of information for  monitoring
trends, sentiments and topics, 
%\cite{ramage2010characterizing}, 
but to
this date not a lot is known about Twitter users.  The majority of studies
regarding users focuses on issues such as the demographics of Twitter, 
%(including methods for inferring sex, age or location information)
%\cite{davisjr:2011}, 
community detection %\cite{java2007we} 
and measures to assess user influence.  However, only a 
few works have focused on another very interesting issue: user profiling\cite{abel:2011}. 

A user profile may contain different elements, including user preferences (as
expressed in messages), personal information (such as gender and location) or
social network behavior (usage time and number of posts per day, among others).
In this work we define users' profiles based on their general preferences,
where a preference is defined by a \textit{semantic topic} discussed in their
posts. 
 
In contrast with previous methods for creating user profiles for specific user
accounts \cite{pennacchiotti2011democrats,abel2011semantic}, here we are
interested in generating profiles of \textit{groups of users}.  These groups
comprise users that share at least one common interest, which may become
explicit when they follow the same Twitter account or frequently discuss the
same topic. Identifying such groups and profiles is of great value for
companies, which can take advantage of them for targeted marketing,
recommendation, or to better understand users preferences over time. 

By profiling groups of users we avoid problems of privacy and data scarcity
while being able to determine the underlying habits and preferences
\cite{zheleva:2009}.  On the other hand, there are several challenges in
such profiling task. The first challenge arises when determining the
semantic topics that characterize the users, since they must be,
simultaneously, the \textit{smallest representative} set of \textit{highly
cohesive} and \textit{non-fragmented} topics observed in a set of messages.
The second challenge is related to the efficiency of the profiling method,
since the number of posts is usually huge and increasing. The third challenge
is the lack of context associated with microblog messages such as Twitter,
making the topic detection, and thus profiling particularly challenging.
The fourth challenge is, given a set of semantic topics, to identify the 
users that will compose each group, coupling with their diversity
and the complexity of the topic. Finally, we may summarize the profile of
each group based on the semantic topics that arise in the group and 
their intensity.

Formally, considering that we extract semantic topics from users posts, any set
of users $U$ can be characterized by a weighted set of preferences
$<$$p_1$,$p_2$,$\ldots$,$p_n$$>$, where $p_x$ represents a semantic topic
identified from the posts of the group. Although we focus just on topics as
preferences in this work, other personal user characteristics may be added to
the preference set later.

In order to build semantic profiles of groups of users, we propose the UPsCAle
(User Profile CreAtor) framework, which is organized in three main phases: (i)
it identifies semantic sub-topics of a user group using a traditional matrix
factorization method; (ii) it merges semantic sub-topics into more cohesive and
unique semantic topics; (iii) it maps the final set of semantic topics into
users profiles.  In the first phase, we benefit from the fact that the task has
been extensively studied~\cite{pons2007topic}, and employ a Non-negative Matrix
Factorization (NMF) \cite{berry2007algorithms}, which is capable of generating
good-quality topics, despite vocabulary overlaps.  However, these semantic
topics may still lack cohesion\cite{cheng:2013}, and we propose a new method
for phase (ii), based on Markovian theory, for merging topics in order to
generate more cohesive topics.  Finally, the desired profiles are then
generated based on a simple strategy that has shown to perform surprisingly
well.

In summary, this paper has two main contributions. The first, a new generic
semantic topic identification method that can be used in any topic detection
task and optimizes the three contradictory but desirable properties in topic
identification simultaneously: representativeness, cohesion and
non-fragmentation. The method is also scalable, efficient and handles the lack
of context usually inherent to microblogs and other social media platforms.
The second contribution is a simple yet effective method to define users
profiles considering groups of users, represented by a weighted set of
preferences that initially correspond to semantic topics. 

\method was evaluated in two phases. First, we tested the topic identification
method in both text and social network collections where the main topics were
known. Second, we performed a case study with over 50,000 users that are
Twitter followers of Barack Obama, considering more than the 700,000 messages
posted during the American Elections in 2012. The results showed that the
method is able to generate less fragmented and more concise topics when
compared to other state of the art methods.% for topics identification.
Such topics enabled the determination of good quality profiles, as the results
show.

%WMJ: do we have anything to say about users?

%The remainder of this paper is organized as follows. Section 2 discusses related work in both user profiles and semantic topic identification. Section 3 introduces UPscALe, the framework developed to find users profiles according to semantic topics. Section 4 describes the experimental methodology, where tests wih both labeled and unlabeled data are performed. Finally, Section 5 draws conclusions and discusses future work directions.

